\section{Part 1: Multi-fidelity trajectories}

In this section, we generate and analyse trajectories from three canonical dynamical systems of increasing complexity. These datasets are employed to evaluate the multi-fidelity Sparse Identification of Nonlinear Dynamics (SINDy) framework.

\subsection{Lorenz System}

The Lorenz system is a three-dimensional autonomous system that exhibits chaotic behaviour for certain parameter regimes. It is defined as:
\begin{equation}
\begin{aligned}
\dot{x} &= \sigma (y - x), \\
\dot{y} &= x(\rho - z) - y, \\
\dot{z} &= xy - \beta z,
\end{aligned}
\label{eq:lorenz}
\end{equation}
where $\sigma$, $\rho$, and $\beta$ are the Prandtl, Rayleigh, and geometric parameters, respectively. In the classical chaotic regime, the parameters are set to $\sigma = 10$, $\rho = 28$, and $\beta = 8/3$.

\begin{figure}
    \includegraphics[width=\textwidth]{Figure/lorenz1.png}
\end{figure}

\begin{figure}
    \begin{center}
    \includegraphics[width=0.65\textwidth]{Figure/lorenzdx.png}
    \includegraphics[width=0.65\textwidth]{Figure/lorenzdy.png}
    \includegraphics[width=0.65\textwidth]{Figure/lorenzdz.png}
    \end{center}
\end{figure}
\subsection{Double Pendulum}

The double pendulum is a canonical example of a nonlinear, chaotic mechanical system consisting of two point masses connected by massless rigid rods. The equations of motion arise from the Lagrangian formulation and can be expressed as:
\begin{equation}
\begin{aligned}
(m_1 + m_2) l_1 \ddot{\theta}_1 + m_2 l_2 \ddot{\theta}_2 \cos(\theta_1 - \theta_2)
+ m_2 l_2 \dot{\theta}_2^2 \sin(\theta_1 - \theta_2)
+ (m_1 + m_2) g \sin \theta_1 &= 0, \\
m_2 l_2 \ddot{\theta}_2 + m_2 l_1 \ddot{\theta}_1 \cos(\theta_1 - \theta_2)
- m_2 l_1 \dot{\theta}_1^2 \sin(\theta_1 - \theta_2)
+ m_2 g \sin \theta_2 &= 0,
\end{aligned}
\label{eq:double_pendulum}
\end{equation}
where $\theta_1$ and $\theta_2$ are the angular displacements of the first and second pendula, $l_1$ and $l_2$ their lengths, $m_1$ and $m_2$ their masses, and $g$ is the gravitational acceleration.

For system identification, it is convenient to express the dynamics by defining angular velocities $\omega_i = \dot{\theta}_i$. The system can then be written as:
\begin{equation}
\begin{aligned}
\dot{\theta}_1 &= \omega_1, \\
\dot{\theta}_2 &= \omega_2, \\
\dot{\omega}_1 &= \frac{
    -g(2m_1 + m_2)\sin\theta_1 - m_2 g \sin(\theta_1 - 2\theta_2)
    - 2 \sin(\theta_1 - \theta_2) m_2 \left(
        \dot{\theta}_2^2 l_2 + \dot{\theta}_1^2 l_1 \cos(\theta_1 - \theta_2)
    \right)
}{
    l_1 \left[ 2m_1 + m_2 - m_2 \cos(2\theta_1 - 2\theta_2) \right]
}, \\
\dot{\omega}_2 &= \frac{
    2 \sin(\theta_1 - \theta_2)
    \left[
        \dot{\theta}_1^2 l_1 (m_1 + m_2)
        + g (m_1 + m_2) \cos\theta_1
        + \dot{\theta}_2^2 l_2 m_2 \cos(\theta_1 - \theta_2)
    \right]
}{
    l_2 \left[ 2m_1 + m_2 - m_2 \cos(2\theta_1 - 2\theta_2) \right]
}.
\end{aligned}
\label{eq:double_pendulum_first_order}
\end{equation}

This formulation highlights the system’s strong nonlinearity and coupling between angular positions and velocities, making it a standard benchmark for data-driven discovery of dynamics.

\begin{figure}
    \begin{center}
        \includegraphics[width=\textwidth]{Figure/doublependulum_multifidelity.png}
    \end{center}
\end{figure}

\begin{figure}
    \begin{center}
        \includegraphics[width=0.33\textwidth]{Figure/heatmap_hf_r2.png}%
        \includegraphics[width=0.33\textwidth]{Figure/heatmap_lf_r2.png}%
        \includegraphics[width=0.33\textwidth]{Figure/heatmap_mf_r2.png}
    \end{center}
\end{figure}
\subsection{Compressible Flow: Isothermal Navier–Stokes}

The one-dimensional isothermal compressible Navier–Stokes equations describe the evolution of a compressible fluid with constant temperature. They are expressed as:
\begin{equation}
\begin{aligned}
\mathbf{u}_t&=-(\mathbf{u} \cdot \nabla) \mathbf{u}+\left(-\nabla P+\mu \nabla^2 \mathbf{u}\right) / \rho, 
\\ \rho_t&=-\nabla \cdot(\rho \mathbf{u})
\end{aligned}
\end{equation}
where $\rho(x,t)$ is the density field, $u(x,t)$ the velocity field, $c$ the isothermal speed of sound, and $\mu$ the dynamic viscosity. The first equation represents conservation of mass, and the second represents conservation of momentum.

These equations can also be non-dimensionalised using reference quantities $\rho_0$, $u_0$, and $L_0$, introducing the Reynolds number $\mathrm{Re} = \rho_0 u_0 L_0 / \mu$ and Mach number $\mathrm{Ma} = u_0 / c$.
