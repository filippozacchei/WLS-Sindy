%=========================================================
% main.tex — Elsevier article template
%=========================================================
\documentclass[final,3p,10.5pt]{elsarticle}
\geometry{margin=1in}
% Journal of the Royal Society
% 2/3 times metting before aps 
% Summary Figure - Overview Figure
% Options: review, 1p, 3p, 5p, twocolumn, authoryear, number

%---------------------------------------------------------
% PACKAGES
%---------------------------------------------------------
\usepackage{amsmath,amssymb,amsfonts}
\usepackage{graphicx}
\usepackage{booktabs}
\usepackage{siunitx}
\usepackage{hyperref}
\usepackage{cleveref}
\usepackage{microtype}
\usepackage{setspace}
\usepackage{xcolor}
\usepackage{natbib}

%---------------------------------------------------------
% BIBLIOGRAPHY STYLE
%---------------------------------------------------------
% Elsevier recommends using biblatex with biber or natbib
\bibliographystyle{unsrt} % or elsarticle-num

%---------------------------------------------------------
% DOCUMENT INFORMATION
%---------------------------------------------------------
% \journal{Journal Name}

\begin{document}

\begin{frontmatter}

\title{Multi-Fidelity SINDy: Sparse Discovery of Nonlinear Dynamical Systems with Fidelity-Weighted Measurements}
% Multi fidelity: ... or non linear modelling 
% Sparse Nonlnear Modelling with multi fidelity data
% Second Title: Multi Fidelity sindy for Multi Modal


\author[aff1]{Filippo Zacchei\corref{cor1}}
\ead{filippo.zacchei@polimi.it}

\author[aff2]{Ana Larranaga}
% \ead{alarra@uw.edu}

\author[aff3]{Attilio Frangi}
% \ead{attilio.frangi@polimi.it}

\author[aff1]{Andrea Manzoni}
% \ead{andrea1.manzoni@polimi.it}

\author[aff4]{Steven L. Brunton}
% \ead{sbrunton@uw.edu}

\cortext[cor1]{Corresponding author}
\address[aff1]{Politecnico di Milano, MOX - Dept. of Mathematics, p.za Leonardo da Vinci, 32, Milano, 20133, Italy}
\address[aff2]{AI Institute in Dynamic Systems, University of Washington, Seattle, USA}
\address[aff3]{Politecnico di Milano, Dept. of Civil and Environmental Engineering, p.za Leonardo da Vinci, 32, Milano, 20133, Italy}
\address[aff4]{Department of Mechanical Engineering, University of Washington, Seattle, Washington 98195, USA}

%---------------------------------------------------------
% ABSTRACT & KEYWORDS
%---------------------------------------------------------
\begin{abstract}
Experimental data are rarely noise-free and often exhibit varying levels of fidelity. 
Measurement uncertainty may differ across repeated observations, sensor apparatus, or even during a single experiment. 
This work addresses the problem of discovering nonlinear dynamical systems from such inhomogeneous data. 
We extend the Sparse Identification of Nonlinear Dynamical Systems (SINDy) framework to handle variable noise levels
by combining Ensemble SINDy and Weak SINDy within a weighted regression formulation derived from weighted least squares. 
The approach not only mitigates the adverse effects of heteroscedastic noise but also shows that repeated, cost-effective,
low-quality measurements can enhance model recovery, achieving performance comparable or superior to reconstructions
based solely on high-fidelity data. A mathematical intuition for the method’s optimality is provided.

The study is organized in two parts. 
The first examines scenarios with multiple trajectories of the same dynamical system, 
each affected by distinct noise levels, highlighting the benefits of their coherent integration. 
The second focuses on single-trajectory data with spatially or temporally varying noise. 
The methodology is validated on several benchmark systems, including the Lorenz attractor,
the double pendulum, the Hopf oscillator, Burgers’ equation, and the Navier–Stokes equations.

The results demonstrate that properly weighted sparse regression transforms measurement heterogeneity 
from a limitation into a source of information, enabling robust and accurate model discovery in complex dynamical systems.
\end{abstract}

% There is a progression 


% \begin{keyword}
% Learning analytics \sep Data science \sep Educational data mining \sep Predictive modeling
% \end{keyword}

\end{frontmatter}

%=========================================================
% INTRODUCTION
%=========================================================
\section{Introduction}
\begin{enumerate}
    \item First Paragraph (Surrogate Models for Digitial Twin MOdel BAsed Engineering)
        \begin{itemize}
            \item Why is this interesting and Important
            \item Why is it Challenging (go in details of what makes it challeging)
            \item Challenges found by previous works
            \item WHat changed for us today
            \item In this work we do
        \end{itemize}
    \item Second PAragraph - Third *Literature*: 50 papers
    \begin{itemize}
        \item Surrogates for Digital twins with multi fidelity data
        \item Difference from standard auto-regressive and using only one surrogate 
        \item Sindy for Surrogate modeling : pros (flexibility for control physics data limits) and lack of multi fidelity
    \end{itemize}
    \item Paragraph 4: what we do, expanding last bullet of first paragraph
\end{enumerate}

\begin{figure}
    \includegraphics[width=\textwidth]{Figure/figure1.png}
\end{figure}

%=========================================================
% METHODS
%=========================================================
\section{Background}
\subsection{Sparse Identification of Dynamical Systems}
\begin{itemize}
    \item STLSQ
    \item Kaptanoglu (Stability) - Loiseau JC (Constrained SR3) for modifying Least Squares
\end{itemize}
\subsection{Ensemble - Weak Sindy}
\subsection{Weighted Least Squares}

\section{Methodology}
% MEthods 

%=========================================================
% RESULTS
%=========================================================
\section{Results}
\section{Part 1: Multi-fidelity trajectories}

In this section, we generate and analyse trajectories from three canonical dynamical systems of increasing complexity. These datasets are employed to evaluate the multi-fidelity Sparse Identification of Nonlinear Dynamics (SINDy) framework.

\subsection{Lorenz System}

The Lorenz system \cite{lorenz1963deterministic} is a three-dimensional autonomous system that exhibits chaotic behaviour for certain parameter regimes. It is defined as:
\begin{equation}
\begin{aligned}
\dot{x} &= \sigma (y - x), \\
\dot{y} &= x(\rho - z) - y, \\
\dot{z} &= xy - \beta z,
\end{aligned}
\label{eq:lorenz}
\end{equation}
where $\sigma$, $\rho$, and $\beta$ are the Prandtl, Rayleigh, and geometric parameters, respectively. In the classical chaotic regime, the parameters are set to $\sigma = 10$, $\rho = 28$, and $\beta = 8/3$.

\subsection{Double Pendulum}

The double pendulum is a well-known example of a nonlinear, chaotic mechanical system consisting of two point masses connected by rigid, massless rods. The dynamics are governed by the following coupled second-order equations:
\begin{equation}
\begin{aligned}
(m_1 + m_2) l_1 \ddot{\theta}_1 + m_2 l_2 \ddot{\theta}_2 \cos(\theta_1 - \theta_2)
+ m_2 l_2 \dot{\theta}_2^2 \sin(\theta_1 - \theta_2)
+ (m_1 + m_2) g \sin \theta_1 &= 0, \\
m_2 l_2 \ddot{\theta}_2 + m_2 l_1 \ddot{\theta}_1 \cos(\theta_1 - \theta_2)
- m_2 l_1 \dot{\theta}_1^2 \sin(\theta_1 - \theta_2)
+ m_2 g \sin \theta_2 &= 0,
\end{aligned}
\label{eq:double_pendulum}
\end{equation}
where $\theta_1$ and $\theta_2$ are the angular displacements of the two pendulums, $l_1$ and $l_2$ are their lengths, $m_1$ and $m_2$ are their masses, and $g$ is the gravitational acceleration.

For numerical simulations, it is often convenient to express this as a first-order system by introducing angular velocities $\omega_i = \dot{\theta}_i$, such that $\dot{\theta}_i = \omega_i$ and $\dot{\omega}_i = f_i(\theta_1, \theta_2, \omega_1, \omega_2)$.

\subsection{Compressible Flow: Isothermal Navier–Stokes}

The one-dimensional isothermal compressible Navier–Stokes equations describe the evolution of a compressible fluid with constant temperature. They are expressed as:
\begin{equation}
\begin{aligned}
\frac{\partial \rho}{\partial t} + \frac{\partial (\rho u)}{\partial x} &= 0, \\
\frac{\partial (\rho u)}{\partial t} + \frac{\partial (\rho u^2 + c^2 \rho)}{\partial x} &= 
\frac{\partial}{\partial x} \left( \mu \frac{\partial u}{\partial x} \right),
\end{aligned}
\label{eq:isothermal_ns}
\end{equation}
where $\rho(x,t)$ is the density field, $u(x,t)$ the velocity field, $c$ the isothermal speed of sound, and $\mu$ the dynamic viscosity. The first equation represents conservation of mass, and the second represents conservation of momentum.

These equations can also be non-dimensionalised using reference quantities $\rho_0$, $u_0$, and $L_0$, introducing the Reynolds number $\mathrm{Re} = \rho_0 u_0 L_0 / \mu$ and Mach number $\mathrm{Ma} = u_0 / c$.

 \nocite{*}


Include Summaries for results for both part 1 and 2
The blocks of the summaries should resemble the flow of the methodology figures
Make independent plots and use keynote for reorganizing
Liuterature review guids presentaiton of the resutls

%=========================================================
% DISCUSSION
%=========================================================
\section{Discussion}
% Combine with results and conclusion

Follow the outline of the introduction, giving more takeways
Limitations (Bias, Different DoFs simulations)
Future Directions 
Discuss implications and limitations. 
Compare your findings with existing work.

%=========================================================
% CONCLUSION
%=========================================================
\section{Conclusion}

Provide a concise summary of the main outcomes and suggestions for future work.

%=========================================================
% REFERENCES
%=========================================================
\bibliography{references}

\end{document}
