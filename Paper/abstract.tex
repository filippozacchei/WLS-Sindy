\begin{abstract}
Experimental data are rarely noise-free and often exhibit varying levels of fidelity. 
Measurement uncertainty may differ across repeated observations, sensor apparatus, or even during a single experiment. 
This work addresses the problem of discovering nonlinear dynamical systems from such inhomogeneous data. 
We extend the Sparse Identification of Nonlinear Dynamical Systems (SINDy) framework to handle variable noise levels
by combining Ensemble SINDy and Weak SINDy within a weighted regression formulation derived from weighted least squares. 
The approach not only mitigates the adverse effects of heteroscedastic noise but also shows that repeated, cost-effective,
low-quality measurements can enhance model recovery, achieving performance comparable or superior to reconstructions
based solely on high-fidelity data. A mathematical intuition for the method’s optimality is provided.

The study is organized in two parts. 
The first examines scenarios with multiple trajectories of the same dynamical system, 
each affected by distinct noise levels, highlighting the benefits of their coherent integration. 
The second focuses on single-trajectory data with spatially or temporally varying noise. 
The methodology is validated on several benchmark systems, including the Lorenz attractor,
the double pendulum, the Hopf oscillator, Burgers’ equation, and the Navier–Stokes equations.

The results demonstrate that properly weighted sparse regression transforms measurement heterogeneity 
from a limitation into a source of information, enabling robust and accurate model discovery in complex dynamical systems.
\end{abstract}
